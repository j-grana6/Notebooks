\documentclass{article}
\usepackage{amsmath}
\usepackage{amssymb}
\begin{document}
\title{Event-Driven Collusion}
\author{Justin Grana}
\maketitle

\section{Introduction}
This write-up is intended to be the ``white paper'' to an
Econometrica, AER, JET...submission.  It will begin with a brief
introduction that outlines a Bertrand game and the notion of
collusion.  It will then briefly outline some of the foundational as
well as more current literature that investigate collusion in a
Bertrand environment.  Finally the introduction will clearly juxtapose
an event-driven Bertrand game with the current literature.

The second section will outline the model in terms of an SFT-Net.  It
omits the fully rigorous specification of the network in favor of an
intuitive explanation with the requisite notation.  Note that this
section also omits the ``one step deviation principal'' for
event-driven games as well as the notion of subgame perfect/sequential
equilibrium.  Basically, all of this involves a ``translation'' of the
idea of subgame perfect/sequential equilibrium in repeated games to an
event-driven framework.  It is a ton of tedious mathematical
boilerplate but conceptually very simple.  (On another note, it might
make sense to describe something like an ``event-driven equilibrium''
that is similar to subgame perfect.  For example, we can say something
like ``If all players' internal state includes a component that
represents the entire history of past observations, then for
all players $i$ and histories $h_i$, $i$ cannot improve his expected
utility by deviating, the strategy profile is an event-driven
equilibrium.  I'm sure there will be some gotchyas but the intuition
is there.)

The third section will present the results and discuss extensions and
future work.

\subsection{Preliminaries}
\textit{This just outlines the basic static and repeated Bertrand
  game.  Can be skipped if you are familiar with ``trigger
  strategies''}  The standard, static Bertrand example is as follows:  Suppose there are $I=\{1, 2\}$ firms.  Each firm has identical per-unit cost $\theta$.  Firm $i$ chooses a price, $p_i$ in $\mathbb{R}$.  Suppose there is a continuum of consumers on the interval $[0,1]$ with reservation price $r>\theta$. Then, player $i$'s utility function is given by 
\begin{equation}
    \Pi_i(p_i, p_j) =  \begin{cases} 
      p_i-\theta & p_i<p_j, p_i<r \\
      \frac{p_i-\theta}{2} & p_i=p_j, p_i<r \\
      0 & \text{ otherwise }
    \end{cases}
\label{eq:profits}
\end{equation}
As is standard, the only equilibrium in this game is when
$p_i=p_j=\theta$.  This is easy to see since any equilibrium with
$p_i,p_j>\theta$, firms have the incentive to undercut one another by
a tiny amount (from here an throughout, I will assume that firm $i$
can undercut a competitor who charges $p_j$ by charging $p_j-\epsilon$
but the profits are still computed as $p_j - \theta$.

Of course, when the game is repeated in discrete time, the result
change.  First, suppose that the Bertrand game is repeated for a known
and finite amount of time, $T$, and each player has common discount
rate $\delta$.  Now, player's strategies are mappings from histories, $h$,
(all observations of past actions) to next action.  Define the
``finite trigger strategy'' for player $i$ as $a_i(h) = $ ``play $r$
in every period unless $j$ does not play $r$.  In that case, play
$\theta$ forever.''  It can be shown that this is \emph{not} a (symmetric)
Nash equilibrium strategy.  To see why, note that in period $T$,
player $i$ will want to undercut $j$ since he does not receive any
future benefit from not undercutting $j$.  Knowing this, in period
$T-1$, $j$ would no longer have an incentive not to undercut $i$.
This logic can be extrapolated back to period $1$ and therefore the
only Nash equilibrium is when both players charge $\theta$. 

Now, suppose the game is repeated infinitely and extend the finite
trigger strategy to an infinite trigger strategy.  We will now show
that under a high enough discount rate, the infinite trigger strategy
is a subgame perfect equilibrium.  To see why, note that $i$'s total
discounted utility under the trigger strategy (played by both players)
is given by $\frac{r}{2(1-\delta)}$.  Note that if $i$ undercuts $j$,
then his expected utility is given by $r$ since he gets the entire
market share in the first period and then nothing after.  Therefore,
$i$ has an incentive to undercut $j$ in the first period iff
$r>\frac{r}{2(1-\delta)}$ or equivalently $\delta<\frac{1}{2}$.  We
can also show that this is subgame perfect since if
$\delta>\frac{1}{2}$, given one player has deviated, it is weakly
dominant for each player to charge $\theta$. 

\subsection{Literature Review}
Recent literature on Bertrand competition has taken a two-pronged
approach.  The first approach is to extend the game of perfect
observability of the other player's price to understand other
phenomenon.  For example, \cite{AB} model a Bertrand game in which
each firm's per-unit cost dynamics are described by a two-state Markov
chain.  They show that there exists an equilibrium  where all
firms charge the reserve price in every period, regardless of their
per-unit cost.  Again, note that the firms' pricing decisions are
fully observable.

Another example with perfectly observable prices is
\cite{differentialresource}.  This game is a differential game in
which player's must also consume a renewable resource.  Also note that
in this game, the demand function is linear and not a reserve price.
Nevertheless, the demand function and the parameters are known exactly
to the players.  The authors then show that a collusive equilibrium
exists (although collusion was not the main focus of their paper).
Again, this is a game with \emph{no} uncertainty.

The second approach in extending Bertrand games is with secret price
cuts.  That is, players do \emph{not} observe the action of the other
player in the previous round.  Instead, after each period stochastic
profits are realized and firms base their continuation strategy off of
such a signal.  For example, \cite{greenold} shows that there exists a
collusive equilibrium in which players play the collusive price if the
total profit of everyone is greater than a certain constant and then
play revert to marginal cost pricing for $T$ periods if the sum total
of profits is below $c$, and then return.  In other words, there is a
finite punishment period in which the players cut prices.  

Other work such as \cite{powercomm} and \cite{commandcoll} show how
there exist information sharing mechanisms that facilitate collusion.
For example, after each period firms may have an incentive to share
their private profit information.  For example, if firms profits are
correlated, each has an incentive to truthfully announce their profits
because if they announce profits that are ``far away'' from their
opponent, the opponent will think they are lying and therefore enter a
punishment phase.  Again, these models assume \emph{private} profit
and pricing information.

Note the difference between the two approaches to the repeated
Bertrand game.  One is where all information is known and observable
(demand, prices, profits, etc) and the other is when firms receive a
signal that is correlated with both demand and the other player's
prices.  That is, firms can choose to undercut the market but they run
the risk of their signal being too strong.  The event-driven framework
falls into the second category of secrete price cuts.  However, the
nature of the ``correlation'' between prices and information is
different.  In the event-driven framework, a player can undercut the
market without generating \emph{any} signal to the other players.  The
reason is that in the event-driven framework, the players observe
their opponents prices subject to a stochastic lag.  That is, if
player $i$ under undercuts the market at time $t$, player $j$ might
not find out about the undercut until time $t+\tau$.  However, neither
$i$ nor $j$ know \emph{exactly} when prices will be made public.
Therefore, it might be optimal for $i$ to undercut the market for a
fixed amount of time and then, after realizing he has not been caught,
return to the collusive price.  In other words, players in the
event-driven Bertrand model do not focus on \textit{what} their
information tells them about the other player's strategies but
\emph{when} they will receive information about other player's
strategies.

Although this will be made clearer in the following section, in the
event-driven Bertrand model, almost always the players do not know
demand or the other player's price.  They only observe signals in
sporadic intervals and there is no restriction that players must learn
of a change immediately after it happens.  In other words, the
event-driven Bertrand game is a new version of secret price cuts
(in which players are not apprised of the exact state of demand or
other players' prices).

%% What is collusion
%% What are some example with observed actions
%% What are some examples without observed actions
%% How is our model different

\section{Model}
 There is a market node whose state space consists of $\{\underbar{r},
\overline{r}\}$ (as well as the prices of each firm's product).  The market
changes states spontaneously at rate $\gamma$. 
Intuitively, there are two reserve prices.  The market alternates
between these reserve prices at rate $\gamma$.  At rate $\lambda_1$,
the market emits a message that travels instantaneously to player
$1$ that informs player $1$ of the reserve price.  This message is
noiseless.  At a rate of $\lambda_2$, the market does the same but the
message is directed to player $2$.

When player, $i=1,2$ receives a message, he must make two decisions.  First,
he must decide what price to charge in $\mathbb{R}^+$.  Secondly, he
can decide whether or not to send a message to player $2$, indicating
what message he just received.  The players are not restricted to
telling the truth and the message to the other player is costless.  In
other words, the players can engage in cheap talk.

The rewards for player $i$ given his price is $p_i$, the other
player's price is $p_j$ and the reserve price is $r$ is given by:
\begin{equation}
    r_i(p_i, p_j) =  \begin{cases} 
      p_i & p_i<p_j, p_i<r \\
      \frac{p_i}{2} & p_i=p_j, p_i<r \\
      0 & \text{ otherwise }
    \end{cases}
\label{eq:rewards}
\end{equation}
In other words, the players split the market if they both charge below
the reward price and charge the same price.  If the players charge two
different prices, the player with the lowest price captures the entire
market.  If player $i$ is undercut or is charging above the reserve
price, then he earns zero reward.  Here I normalize per unit cost to
$0$.  Therefore, the term ``marginal cost pricing'' will be referred
to the case when a firm charges $0$.  

Finally, the market sends messages to player $1$ and player $2$
simultaneously at rate $\mu$ and those messages traverse
instantaneously.  The messages reveal the price of each player's
product currently in the market.  Upon receiving the public message,
each player can decide to either revert to marginal cost pricing or
maintain its current price.

There are a ton of assumptions built into this model.  Without
explaining, I just want to enumerate which ones can be relaxed.
\begin{enumerate}
  \item We can change from perfect substitutes to a case of linear
    demand.
  \item We can change to a case where firms also observe their profits
    instead of the exact state of the market.
  \item The public price message does not have to contain both players'
    price.  In other words, it could be at rate $\mu_1$ player 1's
    price is known publicly and at rate $\mu_2$, player 2's price is
    known publicly.  What is essential (at least for now) is that the
    price revelations are public (i.e. player 1 sometimes learns
    player 2's price and player 2 knows player 1 learned player 2's
    price.)
 \item We can change to a case where instead of public realizations of
   prices, there is a public realization of profits.
 \item We can extend to $N$ firms.
 \item We can extend to Cournot
 \item Upon receiving the public announcement, firms need not be
   limited to only changing their price to marginal cost pricing.
   However, the results will not be affected if this restriction is
   changed but the formal justification will be much more involved.  
\end{enumerate}
All of these assumptions are used to make the initial math easier and
a bit cleaner.  Each new assumption will add parameters and make the
math messier but all (to some degree) are able to be relaxed.

Now, let's define a collusive information sharing strategy as follows:

\noindent \textbf{Truthful information sharing strategy:} Upon receiving a
message that the reserve price is $\underbar{r}$ ($\overline{r}$), inform the other
player that the reserve price is $\underbar{r}$ ($\overline{r}$), set
price equal to $\underbar{r}$.  Upon being told by the other player
that the reserve price is  $\underbar{r}$ ($\overline{r}$), set price
equal to $\underbar{r}$ ($\overline{r}$).  Upon receiving a public
message that the the price of both players in the market is the same,
take no action.  Upon receiving a public message that the prices of
the two firms in the market are \emph{not} the same, set price equal
to $0$ forever

Intuitively, the truthful information sharing strategy is one in which
all information is shared between the firms and the firms charge the
reserve price whenever they are informed of it.  The claim is that for
specific parameter values, there exists a Nash equilibrium in which
both players play the Truthful information sharing strategy.  

\section{Results}
Below is an interactive plot (you need to install the Mathematica
plugin, fairly simple) that illustrates the parameter region in which
the truthful information sharing strategy exists.  The shaded region
is the ``collusive'' region.


As a ``first pass'' result, we see that firms have more incentive when
$\delta$ is low (remember in continuous time, low $\delta$ means more
value placed on future rewards) and the rate of public price
revelation $\mu$, (i.e. the rate of being caught), is low.





\bibliographystyle{plain}
\bibliography{/home/justin/Documents/event_driven_bertrand/bertrandrefs.bib}

\end{document}

